\documentclass[11pt, a4paper, twoside]{article}
\usepackage[utf8]{inputenc}
\usepackage{amssymb}
\usepackage{amsmath}
\usepackage{commath}
\usepackage{hyperref}
\usepackage{graphicx, subfig} % Allows including images
\usepackage{natbib}
\usepackage[left=1cm, right=1cm, top=2cm, bottom=2cm]{geometry}
\usepackage{csquotes} %handle `` and ""
\MakeOuterQuote{"}  %handle opening/closing quotation
\addtolength{\skip\footins}{2pc plus 5pt} %add space between the footnote and the text
\usepackage{listings}
\usepackage{color}
\graphicspath{{../Images/}}

\definecolor{dkgreen}{rgb}{0,0.6,0}
\definecolor{gray}{rgb}{0.5,0.5,0.5}
\definecolor{mauve}{rgb}{0.58,0,0.82}

\title{\vspace{-2cm} \textbf{Modelling COVID-19 spreading in a network (Summary)}} %set the title vertical position
\author{\vspace{2cm} \textbf{Candidate: Riccardo Milocco -- Supervisor: Marco Baiesi}}
\date{\vspace{-3em}\today}

\begin{document}
%\include{cap_Introduction}

\maketitle
\section{Abstract}
The usual simplified description of epidemic dynamics predicts an exponential growth. This is due to the mean field character of the
dynamical equations of the SIR model. However, a recent paper (Thurner S, Klimek P and Hanel R 2020 Proc. Nat. Acad. Sci. 117, 22684) \cite{Thurner22684} showed
that in a network with fixed connectivity, the nodes become infected at a rate that increases linearly rather than exponentially.
Experimental data for COVID-19 seem to validate this approach. In this thesis we plan to study this model by tuning its parameters.
In particular, we monitor the effect induced by a significant presence of hubs in the network.
References used: \cite{Ferguson-CapturingHumanBehaviour}, \cite{SVespignani-EpSpreadSFNets}, \cite{Thurner22684}

\bibliographystyle{plain}
\bibliography{../bib/my_bibliography.bib}

\end{document}
