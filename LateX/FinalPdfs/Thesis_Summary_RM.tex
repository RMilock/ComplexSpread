\documentclass[11pt, a4paper, twoside]{article}
\usepackage[utf8]{inputenc}
\usepackage{subfiles}
\usepackage{amssymb}
\usepackage{amsmath}
\usepackage{commath}
\usepackage{graphicx, subcaption} % Allows including images
\usepackage{natbib}
\usepackage[left=1cm, right=1cm, top=2cm, bottom=2cm]{geometry}
\usepackage{csquotes} %handle `` and ""
\MakeOuterQuote{"}  %handle opening/closing quotation
\addtolength{\skip\footins}{2pc plus 5pt} %add space between the footnote and the text
\usepackage{listings}
\usepackage{color}
\usepackage{hyperref}
\graphicspath{{./Images/}}

\definecolor{dkgreen}{rgb}{0,0.6,0}
\definecolor{gray}{rgb}{0.5,0.5,0.5}
\definecolor{mauve}{rgb}{0.58,0,0.82}

\title{\vspace{-2cm} \textbf{Modelling COVID-19 spreading in a network (Summary)}} %set the title vertical position
\author{\vspace{2cm} \textbf{Candidate: Riccardo Milocco -- Supervisor: Marco Baiesi}}
\date{\vspace{-3em}\today}

\begin{document}
%\include{cap_Introduction}

\maketitle
\section{Abstract}
The usual simplified description of epidemic dynamics predicts an exponential growth. This is due to the mean field character of the
dynamical equations of the SIR model. However, a recent paper (Thurner S, Klimek P and Hanel R 2020 Proc. Nat. Acad. Sci. 117, 22684) \cite{Thurner::NetBasedExpl} showed
that in a network with fixed connectivity, the nodes become infected at a rate that increases linearly rather than exponentially.
Experimental data for COVID-19 seem to validate this approach. In this thesis we plan to study this model by tuning its parameters.
In particular, we monitor the effect induced by a significant presence of hubs in the network.
References used: \cite{Ferguson::CapturingHumanBehaviour}, \cite{VespignaniSatorras2001Epidemic}, \cite{Thurner::NetBasedExpl}

\section{Summary}
At $8 \, \textnormal{May} \, 2020$, none of the affected Covid-19 states have reached "herd immunity" \footnote{The total infected cases at the first peak were $0.3\%$ of the total population, remarkably low with respect to the SARS reported level that are $0.5\%-0.8\%$ of the entire population.}, but still they have reached the "epidemic peak" due to the containment restrictions on social contacts.
The most striking observation is that the epidemic curves, especially when Non-Pharmaceutical Interventions (NPIs) are introduced, exhibit a linear growth for an extended time interval in contrast with the "S-shaped" logistic curve as predicted by the standard compartmental models. The extended of the linear regime depends on the onset of the measures; while for early stages, as it was the case for many countries \cite{Thurner::NetBasedExpl} ($8 \, \textnormal{May} \, 2020$), an exponential growth dominates the spreading of the disease (\autoref{fig:realstates_linearity}).\\
Indeed, infection may occur for two reasons:
\begin{enumerate}
    \item interaction between an infected and a susceptible person;
    \item contact is "intense" (e.g. long, close,...) enough to lead to a disease transmission.
\end{enumerate}
So, the rationale behind the social distancing is that it takes to a reduction of both of these factors.
On the other hand, the standard SIR model assumes that, defining $N, D$ as the total number of individuals in the population and the average number of contacts respectively, there is the same probability that an individual encounters an infected person ("well-mixed population") and all the nodes have the same number of neighbours ($N-1$ or $D$). Therefore, as claimed in \cite{VespignaniSatorras2001Epidemic}, there is the need of studying how the underlying network affects the spreading of a disease, but still no focus was put on the spreading below the epidemic threshold \cite{Thurner::NetBasedExpl}. 


In this thesis, \textit{the main goal} is to grasp the relevant features of a social network when NPIs measure are at work and compare them with respect to the mean-field approach which annihilate the network in favor of the analytical treatment of the SIR model. The final aims would both to find a topology where the disease spread the slowest and a tecnique to characterize the "sub-exponential (linear) regime". Thus, for this purpose, it has been analyzing both the standard model of the graph-theory ("Erdoes-Rényi", "Watts-Strogatz", "Caveman-Model", "Scale-Free") but also more richer structures, i.e. an ad hoc generated Poissonian Small-World Network. Indeed, the lockdown measures force the network of contacts to be sparser, highly differing for a "typical" social netowork where hubs are present. Except for a "Barabasi-Albert" model, the graoh models don't show the high-degree nodes, as they would allow for a faster spread, diminishing the network distance. 
\newpage
An outline of the \textit{past and future steps} is the following.

At first, it has been pruned a fully connected graph of $N$ nodes while changing concurrently $D$ and $\beta$, i.e. the average of nearest neighbors and the transmissibility, to preserve the "epidemic (initial) strength" $R_0 = \frac{\beta \cdot D}{\mu}$ (\autoref{fig::pruning_p0.0}).
In order to appreciate the difference with respect to the "well-mixed" population model, the so-obtained total cases (green) has been compared with the ones recovered in a "mean-field" fashion (in orange)\footnote{Moreover, it has been reported the daily infected of the network (blue) and the the mean-field ones (magenta)}. More precisely, the "mean-field" approximation is a "D-neighbors-mean-field" where every node has $D$ nearest neighbors picked at random on the whole set of nodes. In this way, everyone has the same probability to get in touch with an infected.

Secondarly, it has been introduced the "rewiring probability" $p$ which enables the distant nodes to connect faster, thus, obtaining a Watts-Strogatz (WS) graph with non-zero $p$. In this case, it is possible to recover the "(pseudo) linearity" of the SIR infection, as it is shown in \autoref{fig::pruning_p0.1}, using two similar methods of choosing the spreading parameters $\beta \textnormal{ and } \mu$:
\begin{itemize}
	\item pick a series of $\beta$ and $\mu$ at random;
	\item fixing $\mu \in [0,1]$ and $p=0.1$. Then, as in the previous point, while halving D, the $\beta$ is doubled.
\end{itemize}
For conciseness, in \autoref{fig::pruning_p0.1}, it is reported only the epidemics obtained by the second "pruning procedure".
\newline
Furthermore, a Poissonian "small-world" network has been developed enabling to taking care of degree heterogeneity (i.e. different social contacts), family-clusters and their overlaps, small-world feature, given the fact that "leisure" activities may connect nodes that are diametrically opposite. In particular, by changing the number of the average degree $D$ it is possible to from light ("exponential spreading" for $D\simeq5$) to severe ("linear spreading" for  $D\simeq2.5$) lock-down.
The Poissonian "small-world" network could be realized by imposing, at first, a Poissonian degree sequence on all nodes ("Configuration Model"); then, re-link the nodes to their closest neighboring nodes, i.e. a nearest neighbors rewiring (NNR) procedure is applied. Thus, obtaining a network with poissonian degree distribution and the small-world property.
The (pseudo) linearity obtain with this graph is reported in the \autoref{fig::NNR_Conf_Model}.

Furthermore, the work focused on analyzing the network spreading on a fully-connected cavemen graph, where the families cores could be taken into consideration; and also enabling the possibility of a "long-range" connection via the rewiring probability $p = 0.1$ (cfr. \autoref{fig::Caveman_Model}). Finally for what concerns the different network models, it has been studied the case of "(pseudo) linear" epidemics on a Barabasi-Albert (B-A) Model, i.e. with the presence of hubs (cfr. \autoref{fig::Barabasi_AlbertSIR}).

Moreover, the critical average degree $D_c$ \footnote{$D_c$ as reported in \cite{Thurner::NetBasedExpl} describes a "first-order-phase transition" between exponential and linear growth of an epidemic} , is going to be recovered for the quoted network topologies. 
In particular, by defining $R(t):=(D-n(t))\beta / \mu$ where $n(t)$ is number of "neighboring" nodes that is not possible to infect, i.e. already infected and recovered.
An application of the $R(t) \stackrel{!}{=} 1$ condition for the Poissonian Small-World network drives to \cite{Thurner::NetBasedExpl}
\begin{equation}
	R(t) \stackrel{!}{=} R_{0net} := \frac{D^{2}}{\langle k^2 \rangle - D} \sim 1 \implies D_c = 1 + \frac{2}{(1+p)\lambda}.
\end{equation}
In the case of figure \autoref{fig:PhaseTransition} $D_c \sim 4.85$ and it could be observed the First-Order phase transition of the standard deviation ($SD$) of the new daily infected as a function of the average degree. In particular, for a linear spreading $SD = 0$ since the number of daily cases is constant; while for an exponential growth $SD$ is considerably far from $0$.
The further work would be to extend the $D_c$ formula for the quoted networks. A similar transition could be observed considering how the final number of recovered behaves in function of the $\lambda:= \beta/\mu$ parameter, which could be understood as a "contest" factor since it is the justapposition of trasmissibility and recovery rates (cfr. \autoref{fig:ER_Epid_Threshold}).

Moreover, the definition of $R(t):=(D-n(t))\lambda$ does not depends on $p$. Thus, with equal $R_0$ there could be a local spread or a global one: as for the spreading of the fire, if an agent could "jump" from a side to the other it would start many epidemic clusters, which grows faster than a pathogen being constrained to move locally. So, further refinements on $R(t)$ has to be introduced for this distinctive trait.

A final up-to-date point, would be to use a Graph Neural Network architecture, based on the previously described scenarios and networks, in order the make epidemics forecasting. For this purpose, a relevant reference, is \cite{Davahli::USA_predicting_Covid19}.


\newpage
\section{Relevant Figures}
\begin{figure}[ht]
	\centering
	\includegraphics[width = .8\textwidth]{COVID-RealStates.png}
	\caption{Total Cases per unit of final size outbreak. US shows an explicit linearity; while Austria, choosing for early NPIs, have a immediate bending of the total cases.}
	\label{fig:realstates_linearity}
\end{figure}
\begin{figure}[t]
	\begin{subfigure}{0.45\textwidth}
		\includegraphics[width=\textwidth, height=9cm]{WS_Pruned_AdjMat_N1000_D14.0_p0.0.png}
		\centering
		\caption{Network information for the regular "D-halving" graph}
		\label{fig::adjmat_pruning_p0.0}
	\end{subfigure}
	\begin{subfigure}{0.45\textwidth}
		\includegraphics[width=\textwidth, height=8cm]{WS_Pruned_SIR_p0_9}
		\caption{(pseudo) linearity of the total number of infected for the "halving-D" method with $p = 0$}
		\label{fig::pruning_p0.0}
	\end{subfigure}
	\end{figure}
	
	\begin{figure}[ht]
		\begin{subfigure}{0.45\textwidth}
			\includegraphics[width=\textwidth, height=9cm]{WS_Pruned_AdjMat_N1000_D6.0_p0.1.png}
			\centering
			\caption{Network information for the WS model with $p = 0.1$}
			\label{fig::adjmat_pruning_p0.1}
		\end{subfigure}
		\hfill
		\begin{subfigure}{0.45\textwidth}
			\includegraphics[width=\textwidth, height=8cm]{WS_Pruned_SIR_p0.1_4_2.png}
			\caption{(pseudo) linearity of the total number of infected for the WS model with $p = 0.1$}
			\label{fig::pruning_p0.1}
		\end{subfigure}
	\end{figure}
	
	\newpage
	\begin{figure}[ht]
		\begin{subfigure}{0.45\textwidth}
			\includegraphics[width=\textwidth, height=9cm]{NNR_Conf_Model_AdjMat_N1000_D6_p0.0.png}
			\centering
			\caption{Network information for the "Poissonian" small-world network with $p = 0$}
			\label{fig::adjmat_NNR_Conf_Model}
		\end{subfigure}
		\hfill
		\begin{subfigure}{0.45\textwidth}
			\includegraphics[width=\textwidth, height=8cm]{NNR_R0_1-2_p0.0_27.png}
			\caption{(pseudo) linearity for the nearest neighbors rewiring network with $p = 0$}
			\label{fig::NNR_Conf_Model}
		\end{subfigure}
	\end{figure}
	\begin{figure}[b]
		\begin{subfigure}{0.45\textwidth}
			\includegraphics[width=\textwidth, height=9cm]{Caveman_Model_AdjMat_N1000_D5.0_p0.1.png}
			\centering
			\caption{Network information for the Caveman network model with $p = 0.1$}
			\label{fig::adjmat_Caveman_Model}
		\end{subfigure}
		\hfill
		\begin{subfigure}{0.45\textwidth}
			\includegraphics[width=\textwidth, height=8cm]{CavemanMod_R0_1-3_p0.1_18.png}
			\caption{"(pseudo) linearity" for the Caveman network model with $p = 0.1$}
			\label{fig::Caveman_Model}
		\end{subfigure}
	\end{figure}
	\begin{figure}[t]
		\begin{subfigure}{0.45\textwidth}
			\includegraphics[width=\textwidth, height=9cm]{B-A_Model_AdjMat_N1000_D4.0_p0.0_m2_N0_2.png}
			\centering
			\caption{Network information for the B-A graph with $p = 0$}
			\label{fig::adjmat_B-A_Model}
		\end{subfigure}
		\hfill
		\begin{subfigure}{0.45\textwidth}
			\includegraphics[width=\textwidth, height=8cm]{B-A_Mod_R0_0-1_p0_6.png}
			\caption{"(pseudo) linearity" for the B-A graph with $p = 0$}
			\label{fig::Barabasi_AlbertSIR}
		\end{subfigure}
	\end{figure}


\begin{figure}[ht]
	\begin{subfigure}{0.5\textwidth}
		\includegraphics[width=\textwidth]{NN_Conf_Model_ordp_p0.3_beta0.1_mu0.25.png}
		\centering
		\caption{First Order Phase Transition $D_c \sim 4.85$}
		\label{fig:PhaseTransition}
	\end{subfigure}
	\hfill
	\begin{subfigure}{0.5\textwidth}
		\includegraphics[width=\textwidth]{ER_Epid_Threshold.jpg}
		\centering
		\caption{First Order Phase Transition $\lambda \sim 0.1 \implies R_0 = \lambda * D \sim 1$ where $D = 10$.}
		\label{fig:ER_Epid_Threshold}
	\end{subfigure}
\end{figure}


\clearpage
\bibliographystyle{plain}
\bibliography{bib/bibliogr.bib}

\end{document}
