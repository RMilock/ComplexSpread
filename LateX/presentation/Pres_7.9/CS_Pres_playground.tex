\documentclass[xcolor={dvipsnames}]{beamer}

\setbeamertemplate{frametitle}[default][center]

\mode<presentation> {

% The Beamer class comes with a number of default slide themes
% which change the colors and layouts of slides. Below this is a list
% of all the themes, uncomment each in turn to see what they look like.

%\usetheme{default}
%\usetheme{AnnArbor}
%\usetheme{Antibes}
%\usetheme{Bergen}
%\usetheme{Berkeley}
%\usetheme{Berlin}
%\usetheme{Boadilla}
%\usetheme{CambridgeUS}
%\usetheme{Copenhagen}
%\usetheme{Darmstadt}
%\usetheme{Dresden}
%\usetheme{Frankfurt}
%\usetheme{Goettingen}
%\usetheme{Hannover}
%\usetheme{Ilmenau}
%\usetheme{JuanLesPins}
%\usetheme{Luebeck}
\usetheme{Madrid}
%\usetheme{Malmoe}
%\usetheme{Marburg}
%\usetheme{Montpellier}
%\usetheme{PaloAlto}
%\usetheme{Pittsburgh}
%\usetheme{Rochester}
%\usetheme{Singapore}
%\usetheme{Szeged}
%\usetheme{Warsaw}

% As well as themes, the Beamer class has a number of color themes
% for any slide theme. Uncomment each of these in turn to see how it
% changes the colors of your current slide theme.

%\usecolortheme{albatross}
\usecolortheme{beaver}
%\usecolortheme{beetle}
%\usecolortheme{crane}
%\usecolortheme{dolphin}
%\usecolortheme{dove}
%\usecolortheme{fly}
%\usecolortheme{lily}
%\usecolortheme{orchid}
%\usecolortheme{rose}
%\usecolortheme{seagull}
%\usecolortheme{seahorse}
%\usecolortheme{whale}
%\usecolortheme{wolverine}

%\setbeamertemplate{footline} % To remove the footer line in all slides uncomment this line
%\setbeamertemplate{footline}[page number] % To replace the footer line in all slides with a simple slide count uncomment this line

%\setbeamertemplate{navigation symbols}{} % To remove the navigation symbols from the bottom of all slides uncomment this line
}

\usepackage{graphicx} % Allows including images
\usepackage{booktabs} % Allows the use of \toprule, \midrule and \bottomrule in tables
\usepackage{animate}
\usepackage{listings}
\usepackage{subcaption}
\usepackage{multirow}
\usepackage{xcolor}
\usepackage{tabularx}
\usepackage{lipsum}
\usepackage{tikz}
%\usepackage{subfig}
%\captionsetup[subfloat]{captionskip=1pt}

%%%%%%%%%%%%%%%%%%%%%%%%%%%%%%%%%%%%%%%%%%%%%%%%%%%
%paths:
\graphicspath{
%{"./Plots/2_WS_Epids_zip(beta,mu)/p0.0/SIR/R0s/gif/"}
%{"./Plots/2_WS_Epids_zip(beta,mu)/p0.1/SIR/R0s/gif/"}
%{"./Plots/Fixed_Plots/"}
%{"./Plots/3.1_NNR_Conf_Model/AdjMat/"}
%{"./Plots/3.1_NNR_Conf_Model/SIR/"}
%{"./Plots/3.1_NNR_Conf_Model/SIR/R0_1-2/R0_flow/"}
%{"./Plots/3.1_NNR_Conf_Model/SIR/R0_2-3/"}
%{"./Plots/3.2_Overlapping_Rew_Add_True_13.5_done/p0.0/AdjMat/"}
%{"./Plots/3.2_Overlapping_Rew_Add_True_13.5_done/p0.0/SIR/"}
%{"./Plots/3.2_Overlapping_Rew_Add_True_13.5_done/p0.1/AdjMat/"}
%{"./Plots/3.2_Overlapping_Rew_Add_True_13.5_done/p0.1/SIR/"}
%{"./Plots/3.2_Overlapping_Rew_Add_True_13.5_done/p0.0/SIR/R0_0-2/"}
%{"./Plots/3.2_Overlapping_Rew_Add_True_13.5_done/p0.0/SIR/R0_2-3/"}
%{"./Plots/3.2_Overlapping_Rew_Add_True_13.5_done/p0.0/SIR/R0_3-5/"}
{"./Plots/"}
}

\newcommand\Wider[2][3em]{%
\makebox[\linewidth][c]{%
  \begin{minipage}{\dimexpr\textwidth+#1\relax}
  \raggedright#2
  \end{minipage}%
  }%
}

%\addtobeamertemplate{headline}{}{%
%\begin{tikzpicture}[remember picture,overlay]
%\node at([shift={(.25\paperwidth,-.42)}]current page.north) {\includegraphics[width=1cm]{unipd_logo}};
%\end{tikzpicture}}

%----------------------------------------------------------------------------------------
%	TITLE PAGE
%----------------------------------------------------------------------------------------

\title[Network Covid-19]{Modeling COVID-19 Spreading In A Network} % The short title appears at the bottom of every slide, the full title is only on the title page

\author{Riccardo Milocco} % Your name
\institute[DFA - UniPD] % Your institution as it will appear on the bottom of every slide, may be shorthand to save space
{
{\normalsize Prof./Dr. Marco Baiesi} \\
\vspace{1cm}
{\normalsize Università Degli Studi di Padova} \\ % Your institution for the title page
\medskip
{\normalsize Dipartimento di Fisica e Astronomia G. Galilei} % Your email address
}
\logo{
	\begin{tikzpicture}[overlay,remember picture]
		\node[left = 0cm] at (current page.330){
			\includegraphics[width=1cm, trim = {0 0 10cm 0}, clip]{unipd_logo}
		};
		\end{tikzpicture}
}
\date{25.10.2021} % Date, can be changed to a custom date

\begin{document}

%\addtobeamertemplate{headline}{}{%
%\begin{tikzpicture}[remember picture,overlay]
%\node at([shift={(.25\paperwidth,-.42)}]current page.north) {\includegraphics[height=3cm,width=2cm]{unipd_logo.png}};
%\end{tikzpicture}}

\begin{frame}
\titlepage % Print the title page as the first slide
\end{frame}

\begin{frame}
\frametitle{Overview} % Table of contents slide, comment this block out to remove it
\tableofcontents % Throughout your presentation, if you choose to use \section{} and \subsection{} commands, these will automatically be printed on this slide as an overview of your presentation
\end{frame}

%----------------------------------------------------------------------------------------
%	PRESENTATION SLIDES
%----------------------------------------------------------------------------------------
\section{Social Networks \& Results}
\begin{frame}{Social Networks \& Results}
	\vspace{-1mm}
	\Wider[2em]
	{
	\begin{figure}[t]
		\begin{subfigure}[t]{0.48\linewidth}
			\centering
			\includegraphics[trim = {3cm, 3cm, 3cm, 3cm}, clip, width = .65\linewidth]{NewPlots/NetworksExhibition/Regular Lattice}
			%\caption{Order Parameter dependence as in the upper plot, but $ p = 0.3$. The fuse model approximation yields a proper estimate of $ D_c \sim 8.3$. The introduction of distant node ease the exponential growth. Hence, $ D_c$ is lower than the one obtained in the $ p = 0$ case.}
		\end{subfigure}
		\hfill
		\centering
		\begin{subfigure}[t]{0.48\linewidth}
			\centering
			\includegraphics[height = .48\textheight]{NewPlots/NetworksExhibition/Overlapping_PSW_new.png}
			%\caption{Order Parameter for Regular OPSW with different recovery rates at $p = 0.3$.
			%The critical degree of a fuse-model is corrected with the rewiring probability $ p$. Thus, the estimates is more similar to the real change of regime.}
		\end{subfigure}
		%\caption{Order Parameter ($ AVG(SD(Cases)) $) for Long-Range OPSW of different pathogens and recovery rates. We plotted both the order parameter and $D$ with errorbars.}
	\end{figure}
	\vspace{-.8cm}
	\begin{figure}[t]
		\begin{subfigure}[t]{0.48\linewidth}
			\centering
			\includegraphics[trim = {1cm 1cm 1cm 3cm}, clip, height = .48\textheight]{NewPlots/NetworksExhibition/Caveman_Model}
		\end{subfigure}
		\hfill
		\begin{subfigure}[t]{0.48\linewidth}
			\centering
			\includegraphics[height = .48\textheight, trim = {0 0 2cm 0}, clip,]{NewPlots/NetworksExhibition/Barabasi_Albert_Model_new.png}
		\end{subfigure}
	\end{figure}
	\vspace{-1cm}
	}
\end{frame}


\begin{frame}{Watts-Strogatz model}
	\begin{figure}
		\begin{subfigure}{.48\linewidth}
			\includegraphics[width = \linewidth, trim = {3cm 1cm 2cm 1cm}, clip]{NewPlots/WS_Pruned/AdjMat/WS_Pruned_AdjMat_1000_500.0_0.0.png}
		\end{subfigure}
		\hspace{1mm}
		\begin{subfigure}{.48\linewidth}
			\includegraphics[width = \linewidth, trim = {3cm 1cm 2cm 1cm}, clip]{NewPlots/WS_Pruned/AdjMat/WS_Pruned_AdjMat_1000_500.0_0.3.png} %trim = {0 24.9cm 0 6cm}
		\end{subfigure}%New_Plots/WS_Pruned/AdjMat/WS_Pruned_AdjMat_1000_30.0_0.1.png}}
	%\caption{Title for both}
	\caption{Regular lattice and extension with long-range interactions $ p = 0.3$.\\
			From top left: Graph Realization, Adjacency Matrix, Degree Distribution. }
	\end{figure}
\end{frame}

\begin{frame}{\textit{Fixed $R_0$} SIR + Regular Lattice}
	\vspace{-2mm}
	\Wider[3em]{
	\begin{figure}[h]
		\centering
		\animategraphics[label = taylor, width=\linewidth, loop, height = 0.7\textheight, controls]{1}{NewPlots/WS_Pruned/Gifs/WS_Pruned/WS_Pruned_p0_}{1}{3}
	\end{figure}
	}
\end{frame}

\begin{frame}{COVID-19 SIR + Regular Watts-Strogatz model}
	\vspace{-2mm}
	\Wider[2em]{
	\begin{figure}[h]
		\centering
		\animategraphics[width=\linewidth, loop, height = 0.7\textheight, controls]{1}{NewPlots/WS_Epids/WS_Epids_}{2}{5}
	\end{figure}
	}
\end{frame}

\begin{frame}{Poissonian Small-World Networks}
	\begin{figure}
		\begin{subfigure}{.48\linewidth}
			\includegraphics[width = \linewidth, trim = {3cm 1cm 2cm 1cm}, clip]{NewPlots/PoissSW/OPSW/NNO_Conf_Model_AdjMat_1000_25.0_0.0.png}
		\end{subfigure}
		\hspace{1mm}
		\begin{subfigure}{.45\linewidth}
			\includegraphics[width = \linewidth, trim = {3cm 1cm 2cm 1cm}, clip]{NewPlots/PoissSW/OPSW/NN_Conf_Model_AdjMat_1000_21.0_0.0.png} %trim = {0 24.9cm 0 6cm}
		\end{subfigure}%New_Plots/WS_Pruned/AdjMat/WS_Pruned_AdjMat_1000_30.0_0.1.png}}
	\caption{``Overlapping" PSW and ``Sparse" PSW.}
	\end{figure}
\end{frame}

\begin{frame}{COVID-19 SIR + Overlapping PSW}
	\vspace{-2mm}
	\Wider[2em]{
	\begin{figure}[h]
		\centering
		\animategraphics[width=\linewidth, loop, height = 0.7\textheight, controls]{1}{NewPlots/PoissSW/OPSW/Gifs/OPSW_}{1}{6}
	\end{figure}
	}
\end{frame}

\begin{frame}{Order Parameter $ := SD(C)$ }
	\vspace{-2mm}
	\Wider[2em]
	{
	\begin{figure}[t]
		\begin{subfigure}[t]{0.48\linewidth}
			\centering
			\includegraphics[width=\linewidth, height = .45\textheight]{NewPlots/PoissSW/OrdP/NNO_Conf_Model_ordp_p0.0_beta0.015_d14.0.png} 
			\label{fig:Ordp_OPSW_COVID19_D14}
		\end{subfigure}
		\hfill
		\begin{subfigure}[t]{0.48\linewidth}
			\centering
			\includegraphics[width=\linewidth, height = .45\textheight]{NewPlots/PoissSW/OrdP/NNO_Conf_Model_ordp_p0.0_beta0.015_d4.0.png}
			\label{fig:Ordp_OPSW_highmu_COVID19}
		\end{subfigure}
		\label{fig:Ordp_OPSW_COVID19_panel}
	\end{figure}
	\vspace{-1cm}
	\begin{figure}[t]
		\begin{subfigure}[t]{0.48\linewidth}
			\includegraphics[width=\linewidth, height = .45\textheight]{NewPlots/PoissSW/OrdP/NNO_Conf_Model_ordp_p0.3_beta0.015_d14.0.png}
			%\caption{Order Parameter dependence as in the upper plot, but $ p = 0.3$. The fuse model approximation yields a proper estimate of $ D_c \sim 8.3$. The introduction of distant node ease the exponential growth. Hence, $ D_c$ is lower than the one obtained in the $ p = 0$ case.}
			\label{fig:Ordp_OPSW_Dc8.3_p0.3}
		\end{subfigure}
		\hfill
		\begin{subfigure}[t]{0.48\linewidth}
			\includegraphics[width=\linewidth, height = .45\textheight]{NewPlots/PoissSW/OrdP/NNO_Conf_Model_ordp_p0.3_beta0.015_d4.0.png}
			%\caption{Order Parameter for Regular OPSW with different recovery rates at $p = 0.3$.
			%The critical degree of a fuse-model is corrected with the rewiring probability $ p$. Thus, the estimates is more similar to the real change of regime.}
			\label{fig:Ordp_OPSW_highmu_p0.3}
		\end{subfigure}
		%\caption{Order Parameter ($ AVG(SD(Cases)) $) for Long-Range OPSW of different pathogens and recovery rates. We plotted both the order parameter and $D$ with errorbars.}
		\label{fig:Ordp_OPSW_COVID19_p0.3_panel}
	\end{figure}
	}
\end{frame}

\begin{frame}{Caveman Model}
	\vspace{-3mm}
	\begin{figure}
		\centering
		\animategraphics[trim = {1cm 1cm 1cm 1cm}, height = .8\textheight, controls]{1}{NewPlots/CavemanModel/AdjMat/CM_AdjMat_}{1}{2}
	\end{figure}
\end{frame}

\begin{frame}{COVID-19 SIR + Caveman Model}
	\vspace{-2mm}
	\begin{figure}[h]
		\centering
		\animategraphics[width=\linewidth, loop, height = 0.7\textheight, controls]{1}{NewPlots/CavemanModel/Gifs/CM_}{1}{4}
	\end{figure}
\end{frame}

\begin{frame}{(Regular) Barabási-Albert Model}
	\vspace{-3mm}
	\begin{figure}
		\centering
		\includegraphics[trim = {1cm 1cm 1cm 0cm}, clip, height = .9\textheight]{NewPlots/BA_Model/AdjMat/BA_Model_AdjMat_1000_28.0_0.0_14_14.png}
	\end{figure}
\end{frame}

\begin{frame}{COVID-19 SIR + Barabási-Albert Model}
	\vspace{-2mm}
	\Wider[2em]{
	\begin{figure}[h]
		\centering
		\animategraphics[width=\linewidth, loop, height = 0.7\textheight, controls]{1}{NewPlots/BA_Model/Gifs/BA_}{2}{3}
	\end{figure}
	}
\end{frame}

\begin{frame}{Order Parameter $:= SD(C)$ }
	\Wider[1em]{
	\begin{figure}
		\begin{subfigure}[t]{.48\linewidth}
			\includegraphics[width = \linewidth]{NewPlots/BA_Model/Ordp/BA_Ordp_1}
		\end{subfigure}
		\hspace{1mm}
		\begin{subfigure}[t]{.48\linewidth}
			\includegraphics[width = \linewidth]{NewPlots/BA_Model/Ordp/BA_Ordp_2.png} %trim = {0 24.9cm 0 6cm}
		\end{subfigure}%New_Plots/WS_Pruned/AdjMat/WS_Pruned_AdjMat_1000_30.0_0.1.png}}
	%\caption{Ord}
	\end{figure}
	}
\end{frame}

\logo{}
\definecolor{darkgreen}{RGB}{0,100,0}
\newcommand{\tabitem}{~~\llap{\textbullet}~~}
\begin{frame}
	\Wider[4em]
	{		
	\begin{center}
		\begin{tabular}{||c|c|c|c||}
			\hline
			\multicolumn{3}{|c|}{Final Outbreak Size} & \multicolumn{1}{c|}{Degree Distribution} \\  
			\hline
			\multicolumn{1}{|c|}{Networks} & \multicolumn{1}{c|}{$ D = 6 $}
			& \multicolumn{1}{c|}{$ D = 8 $} &  \\
			\hline
			\begin{minipage}[c][.22\textheight]{.21\textwidth}
				\centering
				\includegraphics[trim = {1cm 7cm 1cm 4cm}, clip, height = .21\textheight]{NewPlots/NetworksExhibition/Caveman_Model}
			\end{minipage}
			& -- & $\textcolor{darkgreen}{(4.3\pm 1.4 )\%}$ 
			& 
			\begin{tabular}{@{}l@{}}\tabitem Low Heterogeneity \\ \tabitem Clustered Net\end{tabular}
			\\
			\hline
			\begin{minipage}[c][.22\textheight]{.21\textwidth}
				\centering
				\includegraphics[trim = {3cm, 7cm, 3cm, 3.1cm}, clip, height = .21\textheight]{NewPlots/NetworksExhibition/Regular Lattice} 
			\end{minipage}
			& -- & $(6 \pm 2.3 )\%$ & Peaked \\
			\hline
			\begin{minipage}[c][.22\textheight]{.21\textwidth}
				\centering
				\includegraphics[trim = {3cm, 7cm, 3cm, 3.1cm}, clip, height = .21\textheight]{NewPlots/NetworksExhibition/Overlapping_PSW_new.png} 
			\end{minipage}
			& -- & $(7.4 \pm 2.5 )\%$ & \begin{tabular}{@{}c@{}}Mid Heterogeneity \\ (No Hubs)\end{tabular} \\
			\hline
			\begin{minipage}[c][.22\textheight]{.21\textwidth}
				\centering
				\includegraphics[trim = {3cm, 7cm, 3cm, 3.4cm}, clip, height = .21\textheight]{NewPlots/NetworksExhibition/Barabasi_Albert_Model_new.png} 
			\end{minipage}
			& $\textcolor{Red}{(36 \pm 8.9 )\%}$  & -- & \begin{tabular}{@{}c@{}}Highly Heterogeneity \\ (Hubs)\end{tabular}  \\
			\hline
		\end{tabular}
	\end{center}
	}
\end{frame}


\logo{
	\begin{tikzpicture}[overlay,remember picture]
		\node[left = 0cm] at (current page.330){
			\includegraphics[width=1cm, trim = {0 0 10cm 0}, clip]{unipd_logo}
		};
		\end{tikzpicture}
}
\section{Summary and Conclusions}
\begin{frame}{Summary and Conclusions}
\centering
\begin{enumerate}
	\item Used a Watts-Strogatz model to test the \textit{epidemic severity} VS $ R_0$; 
	\item The \underline{peak} of the infected is well estimated \underline{only} for mean-field approximation (yellow star); 
	%\item Introduced a Overlapping PSW $ \Rightarrow$ high final outbreak size;
	%\item \textit{local} Caveman Model with the lowest final outbreak size;
	%\item Barabási-Albert model with the highest final outbreak size due to \textbf{hubs};
	\item \underline{Order} the Social Network models for controlling COVID-19: \textit{local} Caveman,Watts-Strogatz, Poissonian-SW, Barabási-Albert; \\
	\item For Overlapping PSW, $ D_c$ is well estimated by the \textit{Fuse model} \underline{only} for $\beta = 0.015$ and $ d = 14$;
	\item For Barabási-Albert model, estimates of $ D_c$ comes from \textit{homogeneous mean-field};
\end{enumerate}
\end{frame}

\begin{frame}
	\frametitle{}
	\centering
	{\Huge Thank You For Your Attention}
\end{frame}

\end{document}

