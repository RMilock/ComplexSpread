\documentclass[xcolor={dvipsnames}, aspectratio = 43]{beamer}

\setbeamertemplate{frametitle}[default][center]

\mode<presentation> {

% The Beamer class comes with a number of default slide themes
% which change the colors and layouts of slides. Below this is a list
% of all the themes, uncomment each in turn to see what they look like.

%\usetheme{default}
%\usetheme{AnnArbor}
%\usetheme{Antibes}
%\usetheme{Bergen}
%\usetheme{Berkeley}
%\usetheme{Berlin}
%\usetheme{Boadilla}
%\usetheme{CambridgeUS}
%\usetheme{Copenhagen}
%\usetheme{Darmstadt}
%\usetheme{Dresden}
%\usetheme{Frankfurt}
%\usetheme{Goettingen}
%\usetheme{Hannover}
%\usetheme{Ilmenau}
%\usetheme{JuanLesPins}
%\usetheme{Luebeck}
\usetheme{Madrid}
%\usetheme{Malmoe}
%\usetheme{Marburg}
%\usetheme{Montpellier}
%\usetheme{PaloAlto}
%\usetheme{Pittsburgh}
%\usetheme{Rochester}
%\usetheme{Singapore}
%\usetheme{Szeged}
%\usetheme{Warsaw}

% As well as themes, the Beamer class has a number of color themes
% for any slide theme. Uncomment each of these in turn to see how it
% changes the colors of your current slide theme.

%\usecolortheme{albatross}
\usecolortheme{beaver}
%\usecolortheme{beetle}
%\usecolortheme{crane}
%\usecolortheme{dolphin}
%\usecolortheme{dove}
%\usecolortheme{fly}
%\usecolortheme{lily}
%\usecolortheme{orchid}
%\usecolortheme{rose}
%\usecolortheme{seagull}
%\usecolortheme{seahorse}
%\usecolortheme{whale}
%\usecolortheme{wolverine}

%\setbeamertemplate{footline} % To remove the footer line in all slides uncomment this line
%\setbeamertemplate{footline}[page number] % To replace the footer line in all slides with a simple slide count uncomment this line

%\setbeamertemplate{navigation symbols}{} % To remove the navigation symbols from the bottom of all slides uncomment this line

\setbeamertemplate{page number in head/foot}[appendixframenumber] %enable this to have backup slides
}


\usepackage{graphicx} % Allows including images
\usepackage{booktabs} % Allows the use of \toprule, \midrule and \bottomrule in tables
\usepackage{animate}
\usepackage{listings}
\usepackage{subcaption}
\usepackage{xcolor}
\usepackage{tikz}
\usepackage{appendixnumberbeamer}

%%%%%%%%%%%%%%%%%%%%%%%%%%%%%%%%%%%%%%%%%%%%%%%%%%%
%paths:
\graphicspath{
%{"./Plots/2_WS_Epids_zip(beta,mu)/p0.0/SIR/R0s/gif/"}
%{"./Plots/2_WS_Epids_zip(beta,mu)/p0.1/SIR/R0s/gif/"}
%{"./Plots/Fixed_Plots/"}
%{"./Plots/3.1_NNR_Conf_Model/AdjMat/"}
%{"./Plots/3.1_NNR_Conf_Model/SIR/"}
%{"./Plots/3.1_NNR_Conf_Model/SIR/R0_1-2/R0_flow/"}
%{"./Plots/3.1_NNR_Conf_Model/SIR/R0_2-3/"}
%{"./Plots/3.2_Overlapping_Rew_Add_True_13.5_done/p0.0/AdjMat/"}
%{"./Plots/3.2_Overlapping_Rew_Add_True_13.5_done/p0.0/SIR/"}
%{"./Plots/3.2_Overlapping_Rew_Add_True_13.5_done/p0.1/AdjMat/"}
%{"./Plots/3.2_Overlapping_Rew_Add_True_13.5_done/p0.1/SIR/"}
%{"./Plots/3.2_Overlapping_Rew_Add_True_13.5_done/p0.0/SIR/R0_0-2/"}
%{"./Plots/3.2_Overlapping_Rew_Add_True_13.5_done/p0.0/SIR/R0_2-3/"}
%{"./Plots/3.2_Overlapping_Rew_Add_True_13.5_done/p0.0/SIR/R0_3-5/"}
{"./Plots/"}
}

%----------------------------------------------------------------------------------------
%	MY COMMANDS

\newcommand\Wider[2][3em]{%
\makebox[\linewidth][c]{%
  \begin{minipage}{\dimexpr\textwidth+#1\relax}
  \raggedright#2
  \end{minipage}%
  }%
}

\renewcommand{\appendixname}{\texorpdfstring{\translate{Appendix}}{Appendix}} %this for backup slides not counted

\definecolor{darkgreen}{RGB}{0,100,0}
\newcommand{\tabitem}{~~\llap{\textbullet}~~}

%----------------------------------------------------------------------------------------
%	TITLE PAGE
%----------------------------------------------------------------------------------------

\title[COVID-19 in Networks]{Modeling COVID-19 Spreading In A Network} % The short title appears at the bottom of every slide, the full title is only on the title page

\author{Riccardo Milocco} % Your name
\institute[DFA - UniPD] % Your institution as it will appear on the bottom of every slide, may be shorthand to save space
{
{\normalsize Prof./Dr. Marco Baiesi} \\
\vspace{1cm}
{\normalsize Università Degli Studi di Padova} \\ % Your institution for the title page
\medskip
{\normalsize Dipartimento di Fisica e Astronomia G. Galilei} % Your email address
}
\logo{
	\begin{tikzpicture}[overlay,remember picture]
		\node[left = 0cm] at (current page.330){
			\includegraphics[width=1cm, trim = {0 0 10cm 0}, clip]{unipd_logo}
		};
		\end{tikzpicture}
}
\date{25.10.2021} % Date, can be changed to a custom date

\begin{document}

\begin{frame}
\titlepage % Print the title page as the first slide
\end{frame}

\begin{frame}
\frametitle{Overview} % Table of contents slide, comment this block out to remove it
\tableofcontents % Throughout your presentation, if you choose to use \section{} and \subsection{} commands, these will automatically be printed on this slide as an overview of your presentation
\end{frame}

%----------------------------------------------------------------------------------------
%	PRESENTATION SLIDES
%----------------------------------------------------------------------------------------
\section{Motivations}
\begin{frame}{Motivations from COVID-19 Data}
	\begin{figure}[h]
		\centering
		\includegraphics[width = \linewidth]{NewPlots/Motivations/COVID-RealStates_2020-02-25.png}
	\end{figure}
\end{frame}

\section{Goals of the research}
\begin{frame}{Goal: Study a SIR model on networks}
\centering
\begin{itemize}
	\item \underline{Order} the Social Network models (Watts-Strogatz, Caveman, Poissonian-SW, Barabási-Albert) for controlling COVID-19; \vfill
	\item Estimate the time $t_{max}$ of the \underline{peak of the infected} after which the pathogen regresses; \vfill
	\item Propose the \underline{\textit{epidemic severity}} to capture the final fraction of the total cases; \vfill
	\item Introduce an \textit{Order Parameter}:=SD(C) to look for a \underline{I order phase-like transition} among regimes; \vfill
	%\item Estimate the \underline{critical average of social contacts} ($D_c$):
	%\begin{itemize}
	%	\item for $D<D_c$, linear growth at early times;
	%	\vspace{2mm }
	%	\item for $D>D_c$, exponential growth at early times.
	%\end{itemize}
\end{itemize}
\end{frame}

\section{Epidemiology Background: SIR Model}
\begin{frame}{SIR Models}
\vspace{-4mm}
\begin{block}{No Network Structure $\to$ Homogeneous Mean-Field (MF) Model}
	\begin{equation}
		\begin{cases}
			s+r+i = 1 \quad \textnormal{HP: ``closed + \textit{well-mixed} population"}\\ \\
			\frac{ds}{dt} = -\beta D s i, \qquad D:= \langle k \rangle\\ \\ 
			\frac{di}{dt} = \beta D s i - \mu i = \mu( R_0 s - 1) i, \quad R_0:= \frac{\beta D}{ \mu} \equiv D \lambda  \label{eq:SIR_MF}	\\ \\
			\frac{dr}{dt} = +\mu i
		\end{cases}
	\end{equation} 
\end{block}
\begin{block}{Network Structures $\to$ Degree-Based Mean-Field Model}
	\begin{equation}
		\begin{cases}
			\frac{ds_k}{dt} = -\beta  k  s_k \textcolor{Red}{\Theta_k} \quad \textnormal{HP: ``closed population"}\\ \\ 
			\frac{di_k}{dt} = \beta  k  s_k \textcolor{Red}{\Theta_k} - \mu i_k = \beta  k  \, (1-i_k-r_k) \, \textcolor{Red}{\Theta_k} -\mu i_k \label{eqs:SIR_degree-based}
		\end{cases}	
	\end{equation}
\end{block}
\end{frame}

\subsection*{Estimate fraction of total cases}
\begin{frame}
	\frametitle{Fraction of total cases at $t_{max}$}
	\begin{itemize}
		\item Thresholds for Regression in Homogeneous SIR:
		\begin{equation}
			\begin{cases}
				\textcolor{darkgreen}{R_0s(t) \stackrel{!}{=} 1}, \qquad R_0:= \frac{\beta D}{ \mu} \\
				s(t) = 1 - \pi(t)
			\end{cases}
			 \Leftrightarrow  \pi(t_{max}) = 1 - \frac{1}{R_0}
		\end{equation}
		\item Decay Threshold in Heterogenous SIR (at $t \approx 0$):	
		\begin{equation}
			\begin{cases}
				\textcolor{darkgreen}{R_0 \stackrel{!}{=} R_{c-net}:= \frac{D^{2}}{\langle k^2 \rangle - D}}\\ \\
				R(t):=(D-n(t))\lambda \qquad \forall \, t \\ \\
				n(t) \sim 1+(D-1)(1-p)\pi(t)
			\end{cases}
			\Leftrightarrow \pi(t_{c}) = \frac{1}{1-p} 
			\left[1- \frac{R_{c-net}}{R_0-\lambda} \right]
		\end{equation}
	\end{itemize}
\end{frame}

\begin{frame}{Epidemic Severity}
	\begin{block}{Basic Reproduction Number $ R_0$ }
		\begin{equation}
			R_0:= \frac{\beta D}{ \mu}
		\end{equation}
	\end{block}
	\begin{block}{Definition of Epidemic Severity}
		\begin{equation}
			\Delta R_0 (\delta):= \frac{R_0 - R_{c-net}}{\delta}
		\end{equation}
		where 
		\begin{itemize}
			\item $R_{c-net} := \frac{D^{2}}{\textcolor{Red}{\langle k^2 \rangle} - D}.$ Hence, if $\langle k^2 \rangle \uparrow \, \implies \, \Delta R_0 (\delta) \uparrow$;
			\item $\delta:=\langle l \rangle $ is the average path length, e.g. from $1 \leftrightarrow 10$. \\ $ \delta$ increases as the network is less ``weaved togheter''.
		\end{itemize}
	\end{block}
\end{frame}

\begin{frame}
	\frametitle{Example}
	\includegraphics[width=\linewidth]{BPres_30.09/WS_Pruned_SIR_R0_3_N1000_D30.0_p0.0_beta0.007_d14.0}
\end{frame}

\logo{}
\begin{frame}{First Order-like Phase Transition}
	\vspace*{-2mm}
	\begin{block}{\textit{Order Parameter}: standard deviation (SD) of the daily new cases $C$ }
		\begin{itemize}
			\item $\mathbb{O}:=SD(C)$
		\end{itemize} 
	\end{block}
	\vspace{-2mm}
	\begin{block}{Critical average degree $D_c$ for different topologies}
		\begin{itemize}
			\item \textit{Homogenous MF}: $\frac{\beta D_c}{\mu} \stackrel{!}{=} 1 \Leftrightarrow 
			\textcolor{Plum}{D_{c-homog} = \frac{\mu}{\beta}}$
			\item \textit{Erdös-Rényi Degree-Based MF}: 
			\begin{equation}
				\begin{cases}
					n(t) \sim 1, \quad s(t\approx0) \sim 1 \\
					R(t):= (D-n(t))\lambda \stackrel{!}{=} R_{c-net} \sim 1
				\end{cases}
				\Leftrightarrow \textcolor{Blue}{D_{c-ER} = 1+ \frac{ \mu}{\beta}}
			\end{equation}
			\item \textit{Fuse-Model Degree-Based MF}:	
			\begin{equation}
				n(t) \sim 1+(D-1)(1-p)/2 \quad \Leftrightarrow \quad \textcolor{Red}{D_{c-FM} = 1+ \frac{2 \frac{ \mu}{\beta}}{1+p}}
			\end{equation}
		\end{itemize}
	\end{block}
	\begin{tikzpicture}[overlay, remember picture]
		\node[left = 1cm] at (current page.05){
			\includegraphics[width=2cm, trim = {5cm 7cm 4cm 8cm}, clip]{NewPlots/Miscellanea/Erdoes_Renyi_Graph}
		};
	\end{tikzpicture}
	\begin{tikzpicture}[overlay, remember picture]
		\node[left = 2.5cm] at (current page.343){
			\includegraphics[width=2.5cm]{NewPlots/Miscellanea/Fuse_Model1}
		};
	\end{tikzpicture}
\end{frame}

\begin{frame}
	\frametitle{Examples of Order Parameters}
	\begin{figure}[t]
		%\begin{subfigure}[t]{0.52\linewidth}
		%	\vspace{-.5cm}
		%	\centering
		%	\includegraphics[width=\linewidth, keepaspectratio]{NewPlots/PoissSW/OrdP/NNO_Conf_Model_ordp_p0.0_beta0.015_d14.0.png} 
			%\caption{Order Parameter with $D_c \sim 10$ as shown by the small exponential phase at the early times of \autoref{fig:sir_O-PSW_COVID}. The $ D_{c-fuse\, model}$ coming from the fuse model is a proper estimate for the real degree threshold.}
			%\label{fig:Ordp_OPSW_COVID19_D14}
		%\end{subfigure}
		\begin{subfigure}[t]{\linewidth}
			\vspace{-0.5cm}
			\centering
			\includegraphics[width=\linewidth, keepaspectratio]{NewPlots/PoissSW/OrdP/NNO_Conf_Model_ordp_p0.3_beta0.015_d14.0.png}
			%\caption{Order Parameter with higher $D_c$ since recovery rate is enhanced. $D_c \sim 20$.
			%The fuse model deeply overestimates $ D_c$ as the poissonian structure prevents the growth of the standard deviation, i.e. the daily new cases, more than a fuse one. The transition is better described by the ER $ D_c$ .}
			\label{fig:Ordp_OPSW_highmu_COVID19}
		\end{subfigure}
		%\caption{Order Parameter for Regular OPSW with different recovery rates at $p = 0$. }
		\label{fig:Ordp_OPSW_COVID19_panel}
	\end{figure}
\end{frame}

\section{Social Networks \& Results}
\begin{frame}{Social Networks \& Results}
	\vspace{-1mm}
	\Wider[2em]
	{
	\begin{figure}[t]
		\begin{subfigure}[t]{0.48\linewidth}
			\centering
			\includegraphics[height = .48\textheight, trim = {0 0 2cm 0}, clip,]{NewPlots/NetworksExhibition/Barabasi_Albert_Model_new.png}
		\end{subfigure}
		\hfill
		\centering
		\begin{subfigure}[t]{0.48\linewidth}
			\centering
			\includegraphics[height = .48\textheight]{NewPlots/NetworksExhibition/Overlapping_PSW_new.png}
			%\caption{Order Parameter for Regular OPSW with different recovery rates at $p = 0.3$.
			%The critical degree of a fuse-model is corrected with the rewiring probability $ p$. Thus, the estimates is more similar to the real change of regime.}
		\end{subfigure}
		%\caption{Order Parameter ($ AVG(SD(Cases)) $) for Long-Range OPSW of different pathogens and recovery rates. We plotted both the order parameter and $D$ with errorbars.}
	\end{figure}
	\vspace{-0.8cm}
	\begin{figure}[b]
		\begin{subfigure}[b]{0.48\linewidth}
			\centering
			\includegraphics[trim = {3cm, 1cm, 3cm, 3cm}, clip, height = .7\textwidth]{NewPlots/NetworksExhibition/Regular Lattice}
			%\caption{Order Parameter dependence as in the upper plot, but $ p = 0.3$. The fuse model approximation yields a proper estimate of $ D_c \sim 8.3$. The introduction of distant node ease the exponential growth. Hence, $ D_c$ is lower than the one obtained in the $ p = 0$ case.}
		\end{subfigure}
		\hfill
		\begin{subfigure}[b]{0.48\linewidth}
			\centering
			\includegraphics[trim = {1cm 1cm 1cm 3cm}, clip, height = .48\textheight]{NewPlots/NetworksExhibition/Caveman_Model}
		\end{subfigure}
	\end{figure}
	\vspace{-1cm}
	}
\end{frame}

\begin{frame}{Decreasing Heterogeneity}
	\begin{figure}
		\centering
		\animategraphics[trim = {0cm 5.5cm 1cm 4cm}, height = .78\textheight, controls]{1}{/home/hal21/MEGAsync/Tour_Physics2.0/Thesis/NetSciThesis/Project/ComplexSpread/LateX/presentation/Pres_7.9/Plots/NewPlots/HL_Heterog/AdjMat/AdjMat_title_}{1}{4}
	\end{figure}
\end{frame}

\begin{frame}{Decreasing COVID-19 Outbreaks}
	\vspace{-2mm}
	\Wider[2em]{
	\begin{figure}
		\centering
		\animategraphics[trim = {1cm 0cm 1cm 5cm}, width=\linewidth, loop, height = .8\textheight, controls]{1}{/home/hal21/MEGAsync/Tour_Physics2.0/Thesis/NetSciThesis/Project/ComplexSpread/LateX/presentation/Pres_7.9/Plots/NewPlots/HL_Heterog/SIR/SIR_pdf_}{1}{4}
	\end{figure}
	}
\end{frame}

\logo{}
\begin{frame}
	\Wider[4em]
	{		
	\begin{center}
		\begin{tabular}{||c|c|c|c||}
			\hline
			\multicolumn{3}{|c|}{Final Outbreak Size} & \multicolumn{1}{c|}{Degree Distribution} \\  
			\hline
			\multicolumn{1}{|c|}{Networks} & \multicolumn{1}{c|}{$ D = 6 $}
			& \multicolumn{1}{c|}{$ D = 8 $} &  \\
			\hline
			\begin{minipage}[c][.22\textheight]{.21\textwidth}
				\centering
				\includegraphics[trim = {1cm 7cm 1cm 4cm}, clip, height = .21\textheight]{NewPlots/NetworksExhibition/Caveman_Model}
			\end{minipage}
			& -- & $\textcolor{darkgreen}{(4.3\pm 1.4 )\%}$ 
			& 
			\begin{tabular}{@{}l@{}}\tabitem Low Heterogeneity \\ \tabitem Clustered Net\end{tabular}
			\\
			\hline
			\begin{minipage}[c][.22\textheight]{.21\textwidth}
				\centering
				\includegraphics[trim = {3cm, 7cm, 3cm, 3.1cm}, clip, height = .21\textheight]{NewPlots/NetworksExhibition/Regular Lattice} 
			\end{minipage}
			& -- & $(6 \pm 2.3 )\%$ & Peaked \\
			\hline
			\begin{minipage}[c][.22\textheight]{.21\textwidth}
				\centering
				\includegraphics[trim = {3cm, 7cm, 3cm, 3.1cm}, clip, height = .21\textheight]{NewPlots/NetworksExhibition/Overlapping_PSW_new.png} 
			\end{minipage}
			& -- & $(7.4 \pm 2.5 )\%$ & \begin{tabular}{@{}c@{}}Mid Heterogeneity \\ (No Hubs)\end{tabular} \\
			\hline
			\begin{minipage}[c][.22\textheight]{.21\textwidth}
				\centering
				\includegraphics[trim = {3cm, 7cm, 3cm, 3.4cm}, clip, height = .21\textheight]{NewPlots/NetworksExhibition/Barabasi_Albert_Model_new.png} 
			\end{minipage}
			& $\textcolor{Red}{(36 \pm 8.9 )\%}$  & -- & \begin{tabular}{@{}c@{}}Highly Heterogeneity \\ (Hubs)\end{tabular}  \\
			\hline
		\end{tabular}
	\end{center}
	}
\end{frame}


\logo{
	\begin{tikzpicture}[overlay,remember picture]
		\node[left = 0cm] at (current page.330){
			\includegraphics[width=1cm, trim = {0 0 10cm 0}, clip]{unipd_logo}
		};
		\end{tikzpicture}
}

\begin{frame}{\textit{Fixed $R_0$} SIR + Regular Lattice}
	\vspace{-2mm}
	\Wider[3em]{
	\begin{figure}[h]
		\centering
		\animategraphics[width=\linewidth, loop, height = 0.7\textheight, controls]{1}{NewPlots/WS_Pruned/Gifs/WS_Pruned/WS_Pruned_p0_}{1}{3}
	\end{figure}
	}
\end{frame}

\begin{frame}{Order Parameter ($SD(C)$) of an Overlapping PSW}
	\vspace{-2mm}
	\Wider[1em]
	{
	\begin{figure}[t]
		\begin{subfigure}[t]{0.48\linewidth}
			\centering
			\includegraphics[width=\linewidth, height = .45\textheight]{NewPlots/PoissSW/OrdP/NNO_Conf_Model_ordp_p0.0_beta0.015_d14.0.png} 
			%\label{fig:Ordp_OPSW_COVID19_D14}
		\end{subfigure}
		\hfill
		\begin{subfigure}[t]{0.48\linewidth}
			\centering
			\includegraphics[width=\linewidth, height = .45\textheight]{NewPlots/PoissSW/OrdP/NNO_Conf_Model_ordp_p0.0_beta0.015_d4.0.png}
			%\label{fig:Ordp_OPSW_highmu_COVID19}
		\end{subfigure}
		%\label{fig:Ordp_OPSW_COVID19_panel}
	\end{figure}
	\vspace{-1cm}
	\begin{figure}[t]
		\begin{subfigure}[t]{0.48\linewidth}
			\includegraphics[width=\linewidth, height = .45\textheight]{NewPlots/PoissSW/OrdP/NNO_Conf_Model_ordp_p0.3_beta0.015_d14.0.png}
			%\caption{Order Parameter dependence as in the upper plot, but $ p = 0.3$. The fuse model approximation yields a proper estimate of $ D_c \sim 8.3$. The introduction of distant node ease the exponential growth. Hence, $ D_c$ is lower than the one obtained in the $ p = 0$ case.}
			\label{fig:Ordp_OPSW_Dc8.3_p0.3}
		\end{subfigure}
		\hfill
		\begin{subfigure}[t]{0.48\linewidth}
			\includegraphics[width=\linewidth, height = .45\textheight]{NewPlots/PoissSW/OrdP/NNO_Conf_Model_ordp_p0.3_beta0.015_d4.0.png}
			%\caption{Order Parameter for Regular OPSW with different recovery rates at $p = 0.3$.
			%The critical degree of a fuse-model is corrected with the rewiring probability $ p$. Thus, the estimates is more similar to the real change of regime.}
			\label{fig:Ordp_OPSW_highmu_p0.3}
		\end{subfigure}
		%\caption{Order Parameter ($ AVG(SD(Cases)) $) for Long-Range OPSW of different pathogens and recovery rates. We plotted both the order parameter and $D$ with errorbars.}
		\label{fig:Ordp_OPSW_COVID19_p0.3_panel}
	\end{figure}
	}
\end{frame}

\begin{frame}{Order Parameter ($SD(C)$) of a Barabási-Albert Model}
	\Wider[1em]{
	\begin{figure}
		\begin{subfigure}[t]{.48\linewidth}
			\includegraphics[width = \linewidth]{NewPlots/BA_Model/Ordp/BA_Ordp_1}
		\end{subfigure}
		\hspace{1mm}
		\begin{subfigure}[t]{.48\linewidth}
			\includegraphics[width = \linewidth]{NewPlots/BA_Model/Ordp/BA_Ordp_2.png} %trim = {0 24.9cm 0 6cm}
		\end{subfigure}%New_Plots/WS_Pruned/AdjMat/WS_Pruned_AdjMat_1000_30.0_0.1.png}}
	%\caption{Ord}
	\end{figure}
	}
\end{frame}

\section{Summary and Conclusions}
\begin{frame}{Summary and Conclusions}
\centering
\begin{enumerate}
	\item \underline{Order} the Social Network models for controlling COVID-19: \textit{local} Caveman,Regular Lattice, Poissonian-SW, Barabási-Albert; \\
	\item The \underline{peak} of the infected is well estimated \underline{only} for mean-field approximation (yellow star); 
	%\item Introduced a Overlapping PSW $ \Rightarrow$ high final outbreak size;
	%\item \textit{local} Caveman Model with the lowest final outbreak size;
	%\item Barabási-Albert model with the highest final outbreak size due to \textbf{hubs};
	\item Used a Regular Lattice to test the \textit{epidemic severity} VS $ R_0$; 
	\item For Overlapping PSW, $ D_c$ is well estimated by the \textit{Fuse model} \underline{only} for $ p = 0.3$, $\beta = 0.015$ and $ d = 14$;
	\item For Barabási-Albert model, the \textit{homogeneous mean-field} estimates the $ D_c$ for $d = 4$ while for $ d = 14, \, SD(C) \propto D$;
\end{enumerate}
\end{frame}

\begin{frame}
	\frametitle{}
	\centering
	{\Huge Thank You For Your Attention}
\end{frame}

\appendix
\begin{frame}
	\frametitle{Thresholds for Regression}
	\vspace{-1mm}
	\begin{block}{Homogeneous MF Model}
		\begin{equation}
				\frac{di}{dt} = \mu( R_0 s(t) - 1) i(t) < 0 \Leftrightarrow \textcolor{darkgreen}{R_0s(t) < 1}
		\end{equation} 
	\end{block}
	\begin{block}{Degree-Based MF: $t \approx 0 \Rightarrow s,i \approx 1,i_0$ and $k\sim D$ for random net}
		\begin{equation}
			\begin{cases}
				\frac{di_k}{dt} = \mu \left[\frac{\beta k}{\mu} \textcolor{Red}{\Theta_k(t)}  s_k(t) - i_k(t) \right] < 0\\ \\
				\Theta_k(t) = \frac{\sum_{k'} (k'-1)p_{k'}i_{k'}(t)}{\langle k \rangle} =
				\textcolor{black}{i_0 \frac{ D - 1}{D} e^{t/\tau}} = \Theta \\ \\
				\tau := \left[\frac{\beta(\langle k^2 \rangle -D)-D \mu}{D}\right]^{-1} < 0 	\Leftrightarrow 
						\textcolor{darkgreen}{R_0 := \frac{\beta D}{ \mu} < \frac{D^{2}}{\langle k^2 \rangle - D}}
			\end{cases}	
		\end{equation}
	\end{block}
\end{frame}

\begin{frame}{Order Parameter: Erdös-Rényi and Fuse models}
	\begin{figure}
		\begin{subfigure}[c]{.48\linewidth}
			\includegraphics[width = \linewidth, trim = {4cm 1cm 2cm 7.5cm}, clip]{NewPlots/Miscellanea/Erdoes_Renyi_Graph}
			\caption{Erdös-Rényi model ($N = 8, p = 0.4$)}
		\end{subfigure}
		\hspace{0.3mm}
		\begin{subfigure}[c]{.48\linewidth}
			\includegraphics[width = \linewidth, trim = {2cm 2.3cm 1.3cm 0cm}, clip]{NewPlots/Miscellanea/fuse_model} %trim = {0 24.9cm 0 6cm}
			\caption{Fuse model with $N = 13, D = 4$}
		\end{subfigure}%New_Plots/WS_Pruned/AdjMat/WS_Pruned_AdjMat_1000_30.0_0.1.png}}
	%\caption{Title for both}
	\end{figure}
\end{frame}
\end{document}

\begin{frame}{Regular Lattice model}
	\begin{figure}
		\begin{subfigure}{.48\linewidth}
			\includegraphics[width = \linewidth, trim = {3cm 1cm 2cm 1cm}, clip]{NewPlots/WS_Pruned/AdjMat/WS_Pruned_AdjMat_1000_500.0_0.0.png}
		\end{subfigure}
		\hspace{1mm}
		\begin{subfigure}{.48\linewidth}
			\includegraphics[width = \linewidth, trim = {3cm 1cm 2cm 1cm}, clip]{NewPlots/WS_Pruned/AdjMat/WS_Pruned_AdjMat_1000_500.0_0.3.png} %trim = {0 24.9cm 0 6cm}
		\end{subfigure}%New_Plots/WS_Pruned/AdjMat/WS_Pruned_AdjMat_1000_30.0_0.1.png}}
	%\caption{Title for both}
	\caption{Regular lattice and extension with long-range interactions $ p = 0.3$.\\
			From top left: Graph Realization, Adjacency Matrix, Degree Distribution. }
	\end{figure}
\end{frame}



\begin{frame}{COVID-19 SIR + Regular Lattice}
	\vspace{-2mm}
	\Wider[2em]{
	\begin{figure}[h]
		\centering
		\animategraphics[width=\linewidth, loop, height = 0.7\textheight, controls]{1}{NewPlots/WS_Epids/WS_Epids_}{2}{5}
	\end{figure}
	}
\end{frame}

\begin{frame}{Poissonian Small-World Networks}
	\begin{figure}
		\begin{subfigure}{.48\linewidth}
			\includegraphics[width = \linewidth, trim = {3cm 1cm 2cm 1cm}, clip]{NewPlots/PoissSW/OPSW/NNO_Conf_Model_AdjMat_1000_25.0_0.0.png}
		\end{subfigure}
		\hspace{1mm}
		\begin{subfigure}{.45\linewidth}
			\includegraphics[width = \linewidth, trim = {3cm 1cm 2cm 1cm}, clip]{NewPlots/PoissSW/OPSW/NN_Conf_Model_AdjMat_1000_21.0_0.0.png} %trim = {0 24.9cm 0 6cm}
		\end{subfigure}%New_Plots/WS_Pruned/AdjMat/WS_Pruned_AdjMat_1000_30.0_0.1.png}}
	\caption{``Overlapping" PSW and ``Sparse" PSW.}
	\end{figure}
\end{frame}

\begin{frame}{COVID-19 SIR + Overlapping PSW}
	\vspace{-2mm}
	\Wider[2em]{
	\begin{figure}[h]
		\centering
		\animategraphics[width=\linewidth, loop, height = 0.7\textheight, controls]{1}{NewPlots/PoissSW/OPSW/Gifs/OPSW_}{1}{6}
	\end{figure}
	}
\end{frame}



\begin{frame}{Caveman Model}
	\vspace{-3mm}
	\begin{figure}
		\centering
		\animategraphics[trim = {1cm 1cm 1cm 1cm}, height = .8\textheight, controls]{1}{NewPlots/CavemanModel/AdjMat/CM_AdjMat_}{1}{2}
	\end{figure}
\end{frame}

\begin{frame}{COVID-19 SIR + Caveman Model}
	\vspace{-2mm}
	\begin{figure}[h]
		\centering
		\animategraphics[width=\linewidth, loop, height = 0.7\textheight, controls]{1}{NewPlots/CavemanModel/Gifs/CM_}{1}{4}
	\end{figure}
\end{frame}

\begin{frame}{(Regular) Barabási-Albert Model}
	\vspace{-3mm}
	\begin{figure}
		\centering
		\includegraphics[trim = {1cm 1cm 1cm 0cm}, clip, height = .9\textheight]{NewPlots/BA_Model/AdjMat/BA_Model_AdjMat_1000_28.0_0.0_14_14.png}
	\end{figure}
\end{frame}

\begin{frame}{COVID-19 SIR + Barabási-Albert Model}
	\vspace{-2mm}
	\Wider[2em]{
	\begin{figure}[h]
		\centering
		\animategraphics[width=\linewidth, loop, height = 0.7\textheight, controls]{1}{NewPlots/BA_Model/Gifs/BA_}{2}{3}
	\end{figure}
	}
\end{frame}



\end{document}

\end{document}

\section{Results}
\subsection{Watts-Strogatz Model} % Sections can be created in order to organize your presentation into discrete blocks, all sections and subsections are automatically printed in the table of contents as an overview of the talk
%------------------------------------------------
\begin{frame}{The Watts-Strogatz Network}
\begin{figure}
	\begin{subfigure}{.4\linewidth}
		\includegraphics[width = \linewidth, trim = {3cm 1cm 2cm 1cm}, clip]{.png}
	\end{subfigure}
	\vspace{3mm}
	\begin{subfigure}{.6\linewidth}
		\includegraphics[width = \linewidth, trim = {0 0 0 5cm}, height = 6.5cm, clip]{NewPlots/WS_Pruned/AdjMat/WS_Pruned_AdjMat_1000_30.0_0.1.png} %trim = {0 24.9cm 0 6cm}
	\end{subfigure}%New_Plots/WS_Pruned/AdjMat/WS_Pruned_AdjMat_1000_30.0_0.1.png}}
%\caption{Title for both}
\end{figure}
\end{frame}

\end{document}


\subsection{Pruning by zipping D and beta} 
%A subsection can be created just before a set of slides with a common theme to further break down your presentation into chunks

\begin{frame}{\textit{Pruning} Watts-Strogatz SW Network (p = 0.0)}
	\vspace{-2mm}
	\begin{block}{Parameter Selection $\beta, \mu, D, p$ }
		\begin{itemize}
		\item $\beta_{init} \in [0.5 \cdot 10^{-3}]$ and $d = \mu^{-1} = [14,9,6,4,1]$ days fixed
		\item $D \in [1000,500,250,124,62,36,18,8,4,2], p = [0,0.3]$
		\end{itemize} 
	\end{block}
	\begin{figure}[h]
		\centering
		\animategraphics[label = taylor, width=\linewidth,loop,controls]{1}{NewPlots/WS_Pruned/Gifs/WS_Pruned/WS_Pruned_}{1}{3}
	\end{figure}
\end{frame}

\begin{frame}{Adjacency Matrix WS(p = 0.3)}
	\centering
	\includegraphics[width = .8\linewidth, height = \textheight]{NewPlots/WS_Pruned/AdjMat/WS_Pruned_AdjMat_1000_500.0_0.3.png}
\end{frame}

\begin{frame}{WS-SIR Model (p = 0.3)}
	\begin{figure}[h]
		\centering
		\animategraphics[width=\linewidth,loop,controls]{1}{NewPlots/WS_Pruned/Gifs/WS_Pruned_p0.3/WS_Pruned_p0.3_}{1}{3}
	\end{figure}
\end{frame}

\begin{frame}{Watt-Strogatz enhancement for $\beta = \mu \simeq 1.0 \textnormal{ and } p = 0.1$ }
	\centering
	\animategraphics[width=0.7\linewidth, height = 7cm,loop,controls]{1}{/2_WS_Epids_zip(beta,mu)/p0.0/SIR/WS_Enhanc_SIR_p0-1_}{1}{8}
\end{frame}

\subsection{Poissonian SW Network}
\begin{frame}
	\frametitle{Poissonian SW Network}
	\begin{block}{Method}
		\begin{enumerate}
			\item Impose a Poissonian degree sequence on all nodes ("Configuration Model");
			\item Nodes are then linked to their closest neighboring nodes on the circle ("Poissonian SW Network");
			\item With probability $p$ we re-link the links of every node $i$ to a new, randomly chosen target node $j$. This allows for super-spreaders and $D \sim \lambda$;
			\item conventional small-world network with a fixed degree. When results are compared with the Poissonian small-world network, only marginal differences are observed.
		\end{enumerate}
	\end{block}
\end{frame}

\subsection{Network Model Overview}
\begin{frame}{Network Model Overview}
	\begin{figure}[t]
		\vspace*{-.5cm}	
		%: Configurational, Nearest Neighbors (NN) "sparse" Overlap, NN narrow Overlap
		\subfloat["Sparse PSW"]{\includegraphics[width=0.5\linewidth, height = 0.8 \textheight]{NewPlots/PoissSW/SPSW/NN_Conf_Model_AdjMat_1000_49.0_0.0.png}}
		\hfill
		\subfloat["Overlapping PSW"]{\includegraphics[width=0.5\linewidth, height = 0.8 \textheight]{NewPlots/PoissSW/OPSW/NNO_Conf_Model_addE_True_AdjMat_1000_50.0_0.0.png}}
	\end{figure}
\end{frame}

\begin{frame}{SPSW-SIR model (p = 0.0, 0.3)}
	\begin{figure}[h]
		\centering
		\animategraphics[width=\linewidth,height = .75 \textheight, loop,controls]{1}{NewPlots/PoissSW/SPSW/Gifs/SPSW-}{1}{6}
	\end{figure}
\end{frame}

\begin{frame}{OPSW-SIR model (p = 0.0, 0.3)}
	\begin{figure}[h]
		\centering
		\animategraphics[width=\linewidth,height = .75 \textheight, loop,controls]{1}{NewPlots/PoissSW/OPSW/Gifs/OPSW-}{1}{4}
	\end{figure}
\end{frame}

\begin{frame}{Order Parameters - Sparse-Overlapped (p = 0.0, 0.3)}
	\begin{figure}[h]
		\centering
		\animategraphics[width=\linewidth,height = .75 \textheight, loop,controls]{1}{NewPlots/PoissSW/OrdP/OParameter-}{1}{4}
	\end{figure}
\end{frame}

\subsection{Caveman Model}
\begin{frame}{Adjacency Matrix Caveman Model (p = 0.0, 0.3)}
	\begin{figure}[h]
		\centering
		\animategraphics[width = .8 \linewidth, height = .8\textheight, loop, controls]{1}{NewPlots/CavemanModel/AdjMat/CMMod_AdjMat-}{1}{2}
	\end{figure}
\end{frame}

\begin{frame}{SIR Caveman Model (p = 0.0, 0.3)}
	\begin{figure}[h]
		\centering
		\animategraphics[width = \linewidth, height = .8\textheight, loop, controls]{1}{NewPlots/CavemanModel/Gifs/CMod_SIR-}{1}{6}
	\end{figure}
\end{frame}

\subsection{Barabási-Albert Model}

\begin{frame}{Adjacency Matrix Barabási-Albert Model (p = 0.0)}
	\begin{figure}[h]
		\centering
		\animategraphics[width = .8 \linewidth, height = .8\textheight, loop, controls]{1}{NewPlots/BA_Model/AdjMat/BA_AdjMat-}{1}{2}
	\end{figure}
\end{frame}

\begin{frame}{SIR Barabási-Albert Model (p = 0.0)}
	\begin{figure}[h]
		\centering
		\animategraphics[width = \linewidth, height = .8\textheight, loop, controls]{1}{NewPlots/BA_Model/Gifs/BA_SIR-}{1}{3}
	\end{figure}
\end{frame}

\begin{frame}{Order Parameter Barabási-Albert Model (p = 0.0)}
	\begin{figure}[h]
		\centering
		\animategraphics[width = \linewidth, height = .8\textheight, loop, controls]{1}{NewPlots/BA_Model/Ordp/BA_Ordp-}{1}{2}
	\end{figure}
\end{frame}

\end{document}




\subsection{Network Model Overview}
\begin{frame}{The models at p = 0.0, 0.1, 0.2}
\centering
\animategraphics[label = taylor, trim ={0 2cm  0 1cm}, width=0.7\linewidth, height = 7cm,loop,controls]{1}{/1.1_WS_Pruned/AdjMat_Pres/WS_Pruned_AdjMat_p_}{1}{3}
\end{frame}


\begin{frame}{Halving $WS(p = 0.0)$}
	\centering
	\animategraphics[width=0.7\linewidth, height = 7cm,loop,controls]{1}{/1.1_WS_Pruned/p0.0/WS_Pruned_SIR_p0_}{11}{14}
\end{frame}

\begin{frame}{Halving $WS(p = 0.1)$}
	\centering
	\animategraphics[width=0.7\linewidth, height = 7cm,loop,controls]{1}{/1.1_WS_Pruned/p0.1/WS_Pruned_SIR_p0.1_}{9}{14}
\end{frame}	

\begin{frame}{Halving $WS(p = 0.2)$}
	\centering
	\animategraphics[width=0.7\linewidth, totalheight =0.8\textheight,loop,controls]{1}{/1.1_WS_Pruned/p0.2/WS_Pruned_SIR_p0.2_}{8}{9}
\end{frame}

\subsection[Epidemic Scan]{Epidemic Scan} % A subsection can be created just before a set of slides with a common theme to further break down your presentation into chunks

\begin{frame}{\textit{Epidemic Scan} Watts-Strogatz SW Network}
\begin{block}{ \textit{Epidemic Scan} Watts-Strogatz SW Network}
\begin{itemize}
	\item $\beta = [0.01, 1, len = 5]$ and $\mu = [0.01, 1, len = 5]$ fixed
	\item $\beta \textnormal{ and } \mu$ runs independently
	\item $D_{prog} \in [2,4,6,8,10,12,14], p = [0.001] \cup [0.01,0.1, len = 5]$
	\item mean over 200 spreads of the network sir parameters, \newline e.g. $\beta = 0.25, \mu=0.75, D = 12$
	\item retain $R_0 \in [0,6]$ and group by $R_0$
\end{itemize} 
\end{block}
\end{frame}

\begin{frame}{Watt-Strogatz enhancement for $\beta = \mu \simeq 1.0 \textnormal{ and } p = 0.0$ }
	\centering
	\animategraphics[width=0.7\linewidth, height = 7cm,loop,controls]{1}{/2_WS_Epids_zip(beta,mu)/p0.0/AdjMat/WS_PruneEnhanch_Adj_}{1}{4}
\end{frame}


\begin{frame}{Watt-Strogatz enhancement for $\beta = \mu \simeq 1.0 \textnormal{ and } p = 0.1$ }
	\centering
	\animategraphics[width=0.7\linewidth, height = 7cm,loop,controls]{1}{/2_WS_Epids_zip(beta,mu)/p0.0/SIR/WS_Enhanc_SIR_p0-1_}{1}{8}
\end{frame}

\begin{frame}{Nearest Neighbor Parameters}
	\begin{block}{\textit{The Epidemic Parameters}}
		\begin{itemize}
			\item $\beta = [0.01, 1, len = 15]$ and $\mu = [0.01, 1, len = 15]$ fixed
			\item $D_{prog} \in [2,32, len = 16], p = 0$
			\item mean over 200 spreads of the network sir parameters, \newline e.g. $\beta = 0.086, \mu=0.16, D = 12$
			\item retain $R_0 \in [0,6]$ and group by $R_0$
		\end{itemize} 
	\end{block}
\end{frame}

\subsection{NN "sparse" Configurational Model}
\begin{frame}{NN "sparse" CM -- Network}
	\centering
	\animategraphics[trim ={0 2cm  0 0}, width=0.7\linewidth, height = 7cm,loop,controls]{1}{NNR_AdjMat_D2-12-24-32_p0.0_}{1}{4}
\end{frame}
\begin{frame}{NN "sparse" CM}
	\centering
	\animategraphics[trim ={0 2cm  0 0}, width=0.7\linewidth, height = 7cm,loop,controls]{1}{/3.1_NNR_Conf_Model/SIR/NNR_R0_p0_}{1}{8}
\end{frame}

\subsection{NN "narrow" Configurational Model}
\begin{frame}{NN "narrow" CM -- Network with $p = 0.0$}
\begin{figure}[t]
	\vspace*{-.5cm}	
	%: Configurational, Nearest Neighbors (NN) "sparse" Overlap, NN narrow Overlap
	\subfloat{\includegraphics[trim = {0 -1.3cm 0 0}, width=0.5\linewidth, height = 5.5cm]{"Overlapping_Rew_Add_True_AdjMat_N1000_D18.0_p0.0.png"}}
	\hfill
	\subfloat{\includegraphics[trim = {0 -1.3cm 0 0}, width=0.5\linewidth, height = 5.5cm]{"Overlapping_Rew_Add_True_AdjMat_N1000_D123.0_p0.0.png"}}
\end{figure}
\end{frame}

\begin{frame}{NN "narrow" CM -- Network with $p = 0.1$}
	\begin{figure}[t]
		\vspace*{-.5cm}	
		%: Configurational, Nearest Neighbors (NN) "sparse" Overlap, NN narrow Overlap
		\subfloat{\includegraphics[trim = {0 -1.3cm 0 0}, width=0.5\linewidth, totalheight = .8\textheight]{"/3.2_Overlapping_Rew_Add_True_13.5_done/p0.1/AdjMat/Overlapping_Rew_Add_True_AdjMat_N1000_D3.0_p0.1"}}
		\hfill
		\subfloat{\includegraphics[trim = {0 -1.3cm 0 0}, width=0.5\linewidth, totalheight = .8\textheight]{"/3.2_Overlapping_Rew_Add_True_13.5_done/p0.1/AdjMat/Overlapping_Rew_Add_True_AdjMat_N1000_D18.0_p0.1"}}
	\end{figure}
\end{frame}

\begin{frame}{NN "narrow" CM -- SIR with $p = 0.1$}
	\centering
	\animategraphics[trim ={0 2cm  0 0}, width=0.7\linewidth, totalheight = 0.8\textheight,loop,controls]{1}{/home/hal21/MEGAsync/Tour_Physics2.0/Thesis/NetSciThesis/LateX/Presentations/Pres_16.6/Plots/3.2_Overlapping_Rew_Add_True_13.5_done/p0.1/SIR/OON_p0.1_}{1}{8}
\end{frame}

\begin{frame}
	\centering
	\animategraphics[trim ={0 2cm  0 0}, width=0.7\linewidth, totalheight = 0.85\textheight,loop,controls]{1}{/3.2_Overlapping_Rew_Add_True_13.5_done/Sel_ordp/NNO_ordp_}{1}{9}
\end{frame}

\section{Caveman Model}
\subsection{Network Model Overview}
\begin{frame}{Caveman Model Network with $p = 0.0$}
	\begin{figure}[t]
		\vspace*{-.5cm}	
		%: Configurational, Nearest Neighbors (NN) "sparse" Overlap, NN narrow Overlap
		\subfloat{\includegraphics[trim = {0 -1.3cm 0 0}, width=0.5\linewidth, height = 7cm]{"/4_Caveman_Model_15.5_done/p0/AdjMat/Caveman_Model_AdjMat_N999_D8.0_p0"}}
		\hfill
		\subfloat{\includegraphics[trim = {0 -1.3cm 0 0}, width=0.5\linewidth, height = 7cm]{"/4_Caveman_Model_15.5_done/p0.1/AdjMat/Caveman_Model_AdjMat_N999_D8.0_p0.1"}}
	\end{figure}
\end{frame}

\begin{frame}
	\begin{block}{\textit{The Epidemic Parameters}}
		\begin{itemize}
			\item $\beta = [0.001, 1, len = 7]$ and $\mu = [0.001, 1, len = 7]$ fixed
			\item $D_{prog} \in [1,9,len=5], p = 0 \textnormal{ or } 0.1$
			\item mean over 200 spreads of the network sir parameters, \newline e.g. $\beta = 0.001, \mu=0.17, D = 3$
			\item retain $R_0 \in [0,6]$ and group by $R_0$
		\end{itemize} 
	\end{block}
\end{frame}


\begin{frame}{Caveman Model -- SIR with $p = 0.1$}
	\centering
	\animategraphics[trim ={0 2cm  0 0}, width=0.7\linewidth, totalheight = 0.8\textheight,loop,controls]{1}{/4_Caveman_Model_15.5_done/p0.1/SIR/Cmodp01_}{1}{8}
\end{frame}

\section{Barabási-Albert Model}
\subsection{Network Model Overview} 
\begin{frame}{ Network with hub-like spreaders and $p = 0.0$}
	\begin{figure}[t]
		\vspace*{-.5cm}	
		%: Configurational, Nearest Neighbors (NN) "sparse" Overlap, NN narrow Overlap
		\subfloat{\includegraphics[trim = {0 -1.3cm 0 0}, width=0.5\linewidth, totalheight = 0.85\textheight]{"/5_B-A_Model_16.5_done/p0.0/AdjMat/B-A_Model_AdjMat_N1000_D4.0_p0.0_m2_N0_2"}}
		\hfill
		\subfloat{\includegraphics[trim = {0 -1.3cm 0 0}, width=0.5\linewidth, totalheight = 0.85\textheight]{"/5_B-A_Model_16.5_done/p0.0/AdjMat/B-A_Model_AdjMat_N1000_D31.0_p0.0_m16_N0_16"}}
	\end{figure}
\end{frame}

\begin{frame}
	\begin{block}{\textit{The Epidemic Parameters}}
		\begin{itemize}
			\item $N = 1000, D_{prog} \in [2,16,len=8], p = 0$
			\item $\beta = [0.001, 1, len = 15]$ and $\mu = [0.01, 1, len = 13]$ fixed
			
			\item mean over 200 spreads of the network sir parameters, \newline e.g. $\beta = 0.001, \mu=0.17, D = 3$
			\item retain $R_0 \in [0,6]$ and group by $R_0$
		\end{itemize} 
	\end{block}
\end{frame}

\begin{frame}{Barabasi-Albert Model -- $R_0 \in [0,4)\textnormal{ and } p = 0.0$}
	\centering
	\animategraphics[trim ={0 2cm  0 0}, width=0.7\linewidth, height = 7cm,loop,controls]{1}{/5_B-A_Model_16.5_done/p0.0/SIR/BAmodp0_}{1}{3}
\end{frame}


\begin{frame}
	\Huge{\centering{The End}}
\end{frame}

\end{document}

\begin{frame}{The model at $p = [0.001] \cup [0.01,0.1, len = 5]$}
	\centering
	\animategraphics[trim ={0 2cm  0 0}, width=0.7\linewidth, height = 7cm,loop,controls]{1}{1.2_WS_Epids_19.5/Sel_AdjMat/WS_Epids_AdjMat_N1000_}{1}{6}
\end{frame}
	
\begin{frame}{}
	\animategraphics[trim ={0 2cm  0 0}, width=0.7\linewidth, height = 7cm,loop,controls]{1}{/1.2_WS_Epids_19.5/Sel_SIR/WS_Epids_pvar_}{1}{6}
\end{frame}

\begin{frame}{Overlapping poissonian Small-World (O-PSW) Network}
	\begin{figure}
			\includegraphics[width = \linewidth]{Plots/BPres_30.09/NNO_1inf/NNO_Conf_Model_SIR_R0_6_N1000_D27.0_p0.0_beta0.015_d14.0.png}
	\end{figure}
\end{frame}

\begin{frame}{OPSW $p = 0.3$  Network}
	\begin{figure}
		\includegraphics[width = \linewidth]{Plots/BPres_30.09/NNO_1inf/NNO_Conf_Model_SIR_R0_2_N1000_D8.0_p0.3_beta0.015_d14.0.png} %trim = {0 24.9cm 0 6cm}
	\end{figure}
\end{frame}

\section{Curvature}
\begin{frame}{Thesis point}
	\begin{figure}
		\includegraphics[width = \linewidth]{BPres_30.09/Curvatura_Issue.png}
	\end{figure}
\end{frame}

\begin{frame}
	\frametitle{Curvature}
	\begin{block}{Empirical Definition}
		The discrete graph curvature measures how the neighborhoods of a pair of nodes are structurally related. The curvature of an edge $(x, y)$ defines the distance taken to travel from neighbors of x to neighbors of y, compared with the length of edge $(x, y)$.
	\end{block}
	\begin{figure}
		\includegraphics[width = \linewidth]{Plots/BPres_30.09/Curvature_graph_network pdf.png}
	\end{figure}
\end{frame}

\begin{frame}{Appendix point}
	\begin{figure}
		\includegraphics[width = .4\linewidth]{BPres_30.09/FrontWave}
	\end{figure}
	\vspace{-.3cm}
	\begin{figure}
		\includegraphics[width = \linewidth]{BPres_30.09/AppendixThurner}
	\end{figure} \vfill
	\begin{equation*}
		\frac{1-\nu(t)}{\nu(t)} = \frac{v(t+\tau)}{vt} = 1+\frac{\tau}{t} = 1+v \tau \mathcal{K}, \quad \, \mathcal{K} = r^{-1} = (vt)^{-1}
	\end{equation*}
\end{frame}

\begin{frame}
	\frametitle{Work Around}
	Averaging the initial and the final fraction of susceptible, 
	\begin{equation}
		\nu(t_{max}) \sim 1-\frac{vt_{max}+0}{2} = 1/2
	\end{equation}
	where $1 = N$ is the (normalized) total population; $ t_{max}:= N/v = 1/v$ is the maximum time of the infection; $v$ the velocity of propagation of the epidemic front.  
\end{frame}